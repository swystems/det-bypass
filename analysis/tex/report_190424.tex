\documentclass{article}
\usepackage[a4paper, total={6in, 8in}]{geometry}

\begin{document}

\date{04/2024}

\title{\Large \bf Report: communcation jitter analysis, periodic peaks}

\author{Davide Rovelli, Michele Dalle Rive}
\maketitle

\section*{Objective}
We analyse the latency disitribution in the communication between 2 bare-metal 
nodes in a datacenter. The objective of our experiments is to make packet 
processing latency of Linux-based hosts with modern NICs as \textit{deterministic}
as possible. This consists in reducing communication jitter to a minimum, where
$jitter = \max(latency) - \min(latency)$.

Contrarily to usual low tail-latency experiments/systems which aim to reduce the
common-case latency up to some percentile (e.g.$99.5\%, 99.9\%$) we want to minimise 
the $100\%$ latency, even at the cost of (non-substantially) increasing the 
common-case. 

\section*{Methodology}

\subsection*{Benchmark}
We run ping-pong tests across 2 hosts \textbf{A} and \textbf{B} by sending packets
at a fixed sending rate. Each packet carries a packet ID and 4 timestamps:

\begin{itemize}
  \itemsep=-0.8mm
  \item $ts_1$: timestamp just before sending the PING message in host \textbf{A}
  \item $ts_2$: timestamp as soon as the process in host \textbf{B} receives the
  PING packet.
  \item $ts_3$: timestamp just before sending the PONG message in host \textbf{B}
  \item $ts_4$: timestamp as soon as the process in host \textbf{A} receives the
  PONG packet.
\end{itemize}

We then calculate the difference between timestamps of successive packets
to get the variations in latency. For example, $ts_1(33) - ts_1(32)$ where 33 and
32 are the packet IDs, represents the time interval between the PING send timestamp
of packet 32 and the PING send timestamp of packet 33. If there's no jitter,
it should be equal to the sending rate. The difference between $ts_1(33) - ts_1(32)$
and the sending rate represents the relative jitter for that pair. 

\subsection*{Hardware setup}
We use 3 different bare-metal clusters with the following configurations:

\paragraph{Cluster A} 2x CloudLab xl170 nodes

\begin{itemize}
  \itemsep=-0.8mm
  \item CPU:  Intel Xeon E5-2640 v4 at 2.40GHz, 10 cores, 64GB RAM
  \item OS/kernel: Ubuntu 22.04.4 LTS / 6.6.19-060619-generic x86\_64
  \item NIC/driver: Mellanox Connect-X 4 / mlx5 
\end{itemize}

\paragraph{Cluster B} 2x internal cluster nodes: 

\begin{itemize}
  \itemsep=-0.8mm
  \item CPU:  Intel Xeon E5-2680 v4 at 2.40GHz , 28 cores, 64GB RAM
  \item OS/kernel: Ubuntu 22.04.4 LTS / 6.6.19-060619-generic x86\_64
  \item NIC/driver: Mellanox Connect-X 4 / mlx5 
\end{itemize}


Nodes are connected over ethernet, via a TOR switch with zero or minimal network load in order 
to only observe the endhost processing overhead.

\subsection*{Software setup}
We use different \textit{kernel bypass} methods in order to minimize the 
well-known overhead introduced by the classical network stack in Linux machines.
We test the following: 

\begin{itemize}
  \vspace{-0.8mm}
  \itemsep=-0.4mm
  \item RoCE, specifically double-sided SEND/RECV RDMA with the Unreliable Datagram (UD) 
  transport type
  \item eBPF XDP in two different modalities:
  \vspace{-1.8mm}
  \begin{itemize}
    \itemsep=-0.5mm
    \item XDP-poll: packet is stored in a BPF map and polled by the application
    thread
    \item AF\_XDP: XDP socket type which handles TX and RX buffer management to 
    the application
  \end{itemize}
    \item Standard network stack over UDP    
\end{itemize}

\noindent
Packet size: \textit{1024 bytes}


\section*{Analysis of periodic outliers}
Here we show some of the results which include interestingly periodic latency 
outliers, which we refer to as ``peaks''. The objective of this analysis is finding
the root causes of such spikes and possibly eliminating them.

\subsection*{``Arrow-like'' peaks: interrupt coalescence}

\subsection*{Static offset peaks}
In all experiments and all technologies we observe periodic peaks at an offset

\subsection*{Long time-bound peaks}


\end{document}